% !TEX encoding = UTF-8 Unicode
%%% Template originaly created by Karol Kozioł (mail@karol-koziol.net) and modified for ShareLaTeX use
\documentclass[a4paper,11pt]{article}
\usepackage[french]{babel} % English language/hyphenation
\usepackage{mathptmx} % Use the Adobe Times Roman as the default text font together with math symbols from the Sym­bol, Chancery and Com­puter Modern fonts

\usepackage[T1]{fontenc}
\usepackage{ucs}
\usepackage[utf8x]{inputenc}
\usepackage{lipsum} % Inserts dummy text
\usepackage{graphicx}
\usepackage{color}

\renewcommand\familydefault{\sfdefault}
\usepackage{tgheros}
%\usepackage[defaultmono]{droidmono}

\usepackage{amsmath,amsfonts,amssymb,amsthm} % For math equations, theorems, symbols, etc
\usepackage{enumerate}
\usepackage{multicol}
\usepackage{tikz}
\usepackage[tikz]{bclogo}
\usepackage{multicol}
\usepackage{geometry}
\geometry{total={210mm,297mm},
left=25mm,right=25mm,%
bindingoffset=0mm, top=20mm,bottom=20mm}
\usepackage{url}
\usepackage[linkbordercolor=blue]{hyperref}
\author{Ahmed Ammar - IPEST}




\linespread{1.3}

\newcommand{\linia}{\rule{\linewidth}{0.5pt}}

% custom theorems if needed
\newtheoremstyle{mytheor}
    {1ex}{1ex}{\normalfont}{0pt}{\scshape}{.}{1ex}
    {{\thmname{#1 }}{\thmnumber{#2}}{\thmnote{ (#3)}}}

\theoremstyle{mytheor}
\newtheorem{defi}{Definition}

% my own titles
\makeatletter
\renewcommand{\maketitle}{
\begin{center}
\vspace{2ex}
{\huge \textsc{\@title}}
\vspace{1ex}
\\
\linia\\
\@author \hfill \@date
\vspace{4ex}
\end{center}
}
\makeatother
%%%

% custom footers and headers
\usepackage{fancyhdr}
\pagestyle{fancy}
\lhead{}
\chead{}
\rhead{}
\lfoot{}
\cfoot{}
\rfoot{Page \thepage}
\renewcommand{\headrulewidth}{0pt}
\renewcommand{\footrulewidth}{0pt}
%%%%%%%%%%%%%%%%%%%%%%%%%%%%%%%%%%%%%%%%%
%%
%%       Algo
%%%%%%%%%%%%%%%%%%%%%

\usepackage[ruled,vlined, french]{algorithm2e}

%%%%%%%%%%%%%%%%%%%%%%%%%%%%%%%%%%%%%%%%
%%%%%%%%%%%%%%%%%%%%%%%%%%%%%%%%%%%%%%%%%
%%%% CODE
%%%%%%%%%%%%%%%%%%%%%%%%%%%%%%%%%%%%%%%%
\usepackage{fancyvrb} % packages needed for verbatim environments
\usepackage{listingsutf8}

\definecolor{mygreen}{rgb}{0,0.6,0}
\definecolor{mygray}{rgb}{0.5,0.5,0.5}
\definecolor{mymauve}{rgb}{0.58,0,0.82}

\lstset{ 
	backgroundcolor=\color{lightgray!10},   % choose the background color; you must add \usepackage{color} or \usepackage{xcolor}; should come as last argument
	basicstyle=\footnotesize,        % the size of the fonts that are used for the code
	breakatwhitespace=false,         % sets if automatic breaks should only happen at whitespace
	breaklines=true,                 % sets automatic line breaking
	captionpos=b,                    % sets the caption-position to bottom
	commentstyle=\color{mygreen},    % comment style
	deletekeywords={...},            % if you want to delete keywords from the given language
	escapeinside={\%*}{*)},          % if you want to add LaTeX within your code
	extendedchars=true,              % lets you use non-ASCII characters; for 8-bits encodings only, does not work with UTF-8
	firstnumber=1,                % start line enumeration with line 1000
	frameround=fttt,
	frame=single,	                   % adds a frame around the code
	keepspaces=true,                 % keeps spaces in text, useful for keeping indentation of code (possibly needs columns=flexible)
	keywordstyle=\color{blue},       % keyword style
	language=Python,                 % the language of the code
	morekeywords={*,...},            % if you want to add more keywords to the set
	numbers=left,                    % where to put the line-numbers; possible values are (none, left, right)
	numbersep=5pt,                   % how far the line-numbers are from the code
	numberstyle=\tiny\color{mygray}, % the style that is used for the line-numbers
	rulecolor=\color{black},         % if not set, the frame-color may be changed on line-breaks within not-black text (e.g. comments (green here))
	showspaces=false,                % show spaces everywhere adding particular underscores; it overrides 'showstringspaces'
	showstringspaces=false,          % underline spaces within strings only
	showtabs=false,                  % show tabs within strings adding particular underscores
	stepnumber=1,                    % the step between two line-numbers. If it's 1, each line will be numbered
	stringstyle=\color{mymauve},     % string literal style
	tabsize=4,	                   % sets default tabsize to 2 spaces
	title=\lstname                   % show the filename of files included with \lstinputlisting; also try caption instead of title
}


\begin{document}


\title{TD 2 : Instructions itératives}

\maketitle

%%%%%%%%%%%%%%%%%%%%%%%%%%%%%%%%%%%%%%%%%%%%%%%%%%%%%%%%%%%%

\section*{Exercice 1 : Calculer $\pi$ (boucle for)}
Tout au long de l'histoire, les grands esprits ont développé différents schémas de calcul pour le nombre $\pi$. Nous allons ici considérer deux de ces schémas, l'un de Leibniz (1646-1716) et l'autre d'Euler (1707-1783).\\
Le schéma de Leibniz peut être écrit
\begin{equation*}
\pi = 8\sum_{k=0}^{\infty}\frac{1}{(4k + 1)(4k + 3)} ,
\end{equation*}
tandis qu'une forme du schéma d'Euler peut apparaître comme
\begin{equation*}
\pi = \sqrt[]{6\sum_{k=1}^{\infty}\frac{1}{k^2}} .
\end{equation*}
Si seuls les N premiers termes de chaque somme sont utilisés comme approximation de $\pi$, chaque schéma modifié aura calculé $\pi$ avec une certaine erreur.\\
En utilisant des boucles \lstinline|for| :
\begin{itemize}
	\item[\textbf{a.}] Écrivez un programme qui prend N comme entrée de l'utilisateur et calcule la valeur de $\pi$ avec les deux schémas. Exécutez le programme avec N = 100
	\item[\textbf{b.}] Votre programme doit également imprimer l'erreur finale obtenue avec les deux schémas, c'est-à-dire lorsque le nombre de termes est N.
\end{itemize}
\section*{Exercice 2 : Calculer $\pi$ (boucle while)}
Refaire l'\textbf{Exercice 1}, mais en utilisant des boucles \lstinline|while| cette fois-ci.
\section*{Exercice 3 : Aire du rectangle par rapport au cercle}
Considérons un cercle et un rectangle. Le cercle a un rayon $r = 10.6$. Le rectangle a des côtés $a$ et $b$, mais seul $a$ est connu à l'avance. Soit $a = 1.3$ et écrire un programme qui utilise une boucle \lstinline|while| pour trouver le plus grand entier $b$ possible qui donne une aire de rectangle plus petite que l'aire du cercle, mais aussi proche que possible de celle-ci. Exécuter le programme et confirmer qu'il donne la bonne réponse (qui est $b = 271$).

\clearpage

\section*{Exercice 4 : Graphisme en console}
Dans cet exercice, nous utilisons des boucles \lstinline|for| imbriquées.
\begin{itemize}
	\item[\textbf{a.}] Écrire un programme Python qui dessine dans la console le triangle suivant: 
\begin{Verbatim}[frame=leftline, framerule=1.5mm, rulecolor=\color{blue}]
* 
* * 
* * * 
* * * * 
\end{Verbatim}
\item[\textbf{b.}] Écrire un programme Python qui dessine dans la console la pyramide suivante: 
\begin{Verbatim}[frame=leftline, framerule=1.5mm, rulecolor=\color{blue}]
* 
* * 
* * * 
* * * * 
* * * * * 
* * * * 
* * * 
* * 
* 
\end{Verbatim}
\end{itemize}


\end{document}

